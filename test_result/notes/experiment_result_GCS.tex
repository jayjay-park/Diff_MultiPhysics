% Options for packages loaded elsewhere
\PassOptionsToPackage{unicode}{hyperref}
\PassOptionsToPackage{hyphens}{url}
\PassOptionsToPackage{dvipsnames,svgnames,x11names}{xcolor}
%
\documentclass[
]{article}

\usepackage{amsmath,amssymb}
\usepackage{iftex}
\ifPDFTeX
  \usepackage[T1]{fontenc}
  \usepackage[utf8]{inputenc}
  \usepackage{textcomp} % provide euro and other symbols
\else % if luatex or xetex
  \usepackage{unicode-math}
  \defaultfontfeatures{Scale=MatchLowercase}
  \defaultfontfeatures[\rmfamily]{Ligatures=TeX,Scale=1}
\fi
\usepackage{lmodern}
\ifPDFTeX\else  
    % xetex/luatex font selection
    \setmainfont[]{Latin Modern Roman}
  \setmathfont[]{Latin Modern Math}
\fi
% Use upquote if available, for straight quotes in verbatim environments
\IfFileExists{upquote.sty}{\usepackage{upquote}}{}
\IfFileExists{microtype.sty}{% use microtype if available
  \usepackage[]{microtype}
  \UseMicrotypeSet[protrusion]{basicmath} % disable protrusion for tt fonts
}{}
\makeatletter
\@ifundefined{KOMAClassName}{% if non-KOMA class
  \IfFileExists{parskip.sty}{%
    \usepackage{parskip}
  }{% else
    \setlength{\parindent}{0pt}
    \setlength{\parskip}{6pt plus 2pt minus 1pt}}
}{% if KOMA class
  \KOMAoptions{parskip=half}}
\makeatother
\usepackage{xcolor}
\setlength{\emergencystretch}{3em} % prevent overfull lines
\setcounter{secnumdepth}{5}
% Make \paragraph and \subparagraph free-standing
\makeatletter
\ifx\paragraph\undefined\else
  \let\oldparagraph\paragraph
  \renewcommand{\paragraph}{
    \@ifstar
      \xxxParagraphStar
      \xxxParagraphNoStar
  }
  \newcommand{\xxxParagraphStar}[1]{\oldparagraph*{#1}\mbox{}}
  \newcommand{\xxxParagraphNoStar}[1]{\oldparagraph{#1}\mbox{}}
\fi
\ifx\subparagraph\undefined\else
  \let\oldsubparagraph\subparagraph
  \renewcommand{\subparagraph}{
    \@ifstar
      \xxxSubParagraphStar
      \xxxSubParagraphNoStar
  }
  \newcommand{\xxxSubParagraphStar}[1]{\oldsubparagraph*{#1}\mbox{}}
  \newcommand{\xxxSubParagraphNoStar}[1]{\oldsubparagraph{#1}\mbox{}}
\fi
\makeatother


\providecommand{\tightlist}{%
  \setlength{\itemsep}{0pt}\setlength{\parskip}{0pt}}\usepackage{longtable,booktabs,array}
\usepackage{calc} % for calculating minipage widths
% Correct order of tables after \paragraph or \subparagraph
\usepackage{etoolbox}
\makeatletter
\patchcmd\longtable{\par}{\if@noskipsec\mbox{}\fi\par}{}{}
\makeatother
% Allow footnotes in longtable head/foot
\IfFileExists{footnotehyper.sty}{\usepackage{footnotehyper}}{\usepackage{footnote}}
\makesavenoteenv{longtable}
\usepackage{graphicx}
\makeatletter
\def\maxwidth{\ifdim\Gin@nat@width>\linewidth\linewidth\else\Gin@nat@width\fi}
\def\maxheight{\ifdim\Gin@nat@height>\textheight\textheight\else\Gin@nat@height\fi}
\makeatother
% Scale images if necessary, so that they will not overflow the page
% margins by default, and it is still possible to overwrite the defaults
% using explicit options in \includegraphics[width, height, ...]{}
\setkeys{Gin}{width=\maxwidth,height=\maxheight,keepaspectratio}
% Set default figure placement to htbp
\makeatletter
\def\fps@figure{htbp}
\makeatother

\usepackage{arxiv}
\usepackage{orcidlink}
\usepackage{amsmath}
\usepackage[T1]{fontenc}
\makeatletter
\@ifpackageloaded{caption}{}{\usepackage{caption}}
\AtBeginDocument{%
\ifdefined\contentsname
  \renewcommand*\contentsname{Table of contents}
\else
  \newcommand\contentsname{Table of contents}
\fi
\ifdefined\listfigurename
  \renewcommand*\listfigurename{List of Figures}
\else
  \newcommand\listfigurename{List of Figures}
\fi
\ifdefined\listtablename
  \renewcommand*\listtablename{List of Tables}
\else
  \newcommand\listtablename{List of Tables}
\fi
\ifdefined\figurename
  \renewcommand*\figurename{Figure}
\else
  \newcommand\figurename{Figure}
\fi
\ifdefined\tablename
  \renewcommand*\tablename{Table}
\else
  \newcommand\tablename{Table}
\fi
}
\@ifpackageloaded{float}{}{\usepackage{float}}
\floatstyle{ruled}
\@ifundefined{c@chapter}{\newfloat{codelisting}{h}{lop}}{\newfloat{codelisting}{h}{lop}[chapter]}
\floatname{codelisting}{Listing}
\newcommand*\listoflistings{\listof{codelisting}{List of Listings}}
\makeatother
\makeatletter
\makeatother
\makeatletter
\@ifpackageloaded{caption}{}{\usepackage{caption}}
\@ifpackageloaded{subcaption}{}{\usepackage{subcaption}}
\makeatother
\makeatletter
\@ifpackageloaded{sidenotes}{}{\usepackage{sidenotes}}
\@ifpackageloaded{marginnote}{}{\usepackage{marginnote}}
\makeatother

\ifLuaTeX
  \usepackage{selnolig}  % disable illegal ligatures
\fi
\usepackage{bookmark}

\IfFileExists{xurl.sty}{\usepackage{xurl}}{} % add URL line breaks if available
\urlstyle{same} % disable monospaced font for URLs
\hypersetup{
  pdftitle={Data Generation: Two Phase Flow},
  pdfauthor={Jayjay, Tuna, Jason, Richard},
  colorlinks=true,
  linkcolor={blue},
  filecolor={Maroon},
  citecolor={Blue},
  urlcolor={Blue},
  pdfcreator={LaTeX via pandoc}}


\renewcommand{\today}{2024-09-30}
\newcommand{\runninghead}{A Preprint }
\renewcommand{\runninghead}{A Preprint }
\title{Data Generation: Two Phase Flow}
\def\asep{\\\\\\ } % default: all authors on same column
\author{\textbf{Jayjay, Tuna, Jason, Richard}\\}
\date{2024-09-30}
\begin{document}
\maketitle


\section{Surrogate Modeling for Which
System?}\label{surrogate-modeling-for-which-system}

\begin{enumerate}
\def\labelenumi{\arabic{enumi}.}
\tightlist
\item
  Simplified Geological Carbon Storage (Francis' paper)
\item
  Incompressible Navier Stokes
\end{enumerate}

\section{Twophase flow for the CO2
saturation}\label{twophase-flow-for-the-co2-saturation}

\begin{itemize}
\tightlist
\item
  We regenerate Francis' dataset, and additionally compute Fisher
  Information Matrix as well.
\item
  For the purpose of validation, we currently form full Fisher
  Infromation Matrix and then compute eigenvector.
\item
  Our next step will be low rank approximation or trace estimation so
  that we don't have to form the full matrix.
\end{itemize}

\section{Dataset}\label{dataset}

Our dataset consists of \(2000\) pairs of \(\{K, S^t(K)\}_{t=1}^8\).

\begin{figure}

\begin{minipage}{0.50\linewidth}

\centering{

\includegraphics[width=0.8\textwidth,height=\textheight]{../../data/Ks_0.png}

}

\subcaption{\label{fig-surus}K0}

\end{minipage}%
%
\begin{minipage}{0.50\linewidth}

\centering{

\includegraphics[width=0.8\textwidth,height=\textheight]{../../data/Ks_1.png}

}

\subcaption{\label{fig-hanno}K1}

\end{minipage}%

\caption{\label{fig-K}Example Permeability Model}

\end{figure}%

\begin{figure}

\begin{minipage}{\linewidth}

\centering{

\includegraphics[width=1\textwidth,height=\textheight]{../../data/Snew_series.png}

}

\subcaption{\label{fig-S0}Time Series of Saturation of K0}

\end{minipage}%
\newline
\begin{minipage}{\linewidth}

\centering{

\includegraphics[width=1\textwidth,height=\textheight]{../../data/Snew_series1.png}

}

\subcaption{\label{fig-S1}Time Series of Saturation of K1}

\end{minipage}%

\caption{\label{fig-S}Example Saturation Time Series}

\end{figure}%

\section{Fisher Information Matrix}\label{fisher-information-matrix}

\begin{itemize}
\tightlist
\item
  To find the optimal number of observations, \(M\), we visualize
  eigenvector and vector jacobian product.
\item
  Given 1 pair of dataset, \(\{K, S^t(K)\}^8_{t=1}\), we get a single
  FIM.
\end{itemize}

\subsection{Computing Fisher Information Matrix for each
datapoint}\label{computing-fisher-information-matrix-for-each-datapoint}

We consider a realistic scenario when we only have access to samples,
but not distribution. When \(N\) is number of samples and
\(X \in \mathbb{R}^{d \times d}\), neural network model \(F_{nn}\)
learns mapping from \(X_i \rightarrow Y_i\). For each pair of
\(\left\{X_i, Y_i \right\}^N_{i=1}\), we generate
\(\left\{FIM_i\right\}_{i=1}^{N}\).

\begin{itemize}
\tightlist
\item
  \(N\) : number of data points, \(\left\{X_i, Y_i \right\}\)
\item
  \(M\) : number of observation, \(Y\)
\end{itemize}

\begin{quote}
\[ \left\{ X_i \right\}^N_{i=1} \sim p_X(X), \: \epsilon \sim \mathcal{N}(0, \Sigma), \: \Sigma = I
\] For a single data pair, we generate multiple observations.
\[Y_{i, J} = F(X_i) + \epsilon_{i, J}, \quad where \left\{ \epsilon_{i,J}\right\}^{N,M}_{i,J= 1,1}\]
As we assumed Gaussian, we define likelihood as following.
\[p(Y_{i,J}|X_i) = e^{-\frac{1}{2}\|Y_{i,J}-F(X_i)\|^2_2}\]
\[log \: p(Y_{i,J}|X_i) \approx \frac{1}{\Sigma}\|Y_{i,J}-F(X_i)\|^2_2\]
A FIM for a single data pair \(i\) is:
\[FIM_i = \mathbb{E}_{Y_{i, \{J\}^m_{i=1}} \sim p(Y_{i,J}|X_i)} \left[ \left(\nabla log \: p(Y_{i,J}|X_i)\right)\left(\nabla log \: p(Y_{i,J}|X_i)\right)^T\right]\]
\end{quote}

\subsection{\texorpdfstring{When Random Variable of FIM, \(Y\), is both
Saturation and
Pressure}{When Random Variable of FIM, Y, is both Saturation and Pressure}}\label{when-random-variable-of-fim-y-is-both-saturation-and-pressure}

\subsubsection{How does FIM change as number of observation
increases?}\label{how-does-fim-change-as-number-of-observation-increases}

\begin{itemize}
\tightlist
\item
  FIM is expectation of covariance of derivative of log likelihood. As
  we expected, we see clearer definition in diagonal relationship as
  \(M\) increases.
\item
  We observe that as \(M\) increases, the clearer we see the boundary of
  the permeability, which will be more informative during training and
  inference. \sidenote{\footnotesize \href{https://www.overleaf.com/1149716711hxnvfbyfpzvb\#a799ce}{Note
    on Learning Problem}.}
\end{itemize}

\begin{figure}

\begin{minipage}{0.33\linewidth}

\includegraphics[width=1\textwidth,height=\textheight]{../../data/FIM/FIM0_sub0.png}

\subcaption{\label{}M = 1}
\end{minipage}%
%
\begin{minipage}{0.33\linewidth}

\includegraphics[width=1\textwidth,height=\textheight]{../../data/FIM/FIM0_sub0_multi_10.png}

\subcaption{\label{}M = 10}
\end{minipage}%
%
\begin{minipage}{0.33\linewidth}

\includegraphics[width=1\textwidth,height=\textheight]{../../data/FIM/FIM0_sub0_multi_100.png}

\subcaption{\label{}M = 100}
\end{minipage}%

\caption{\label{fig-fim}Change in FIM{[}:256, :256{]} of single data
pair \(\{K, S^t(K)\}^8_{t=1}\) as number of observation, \(M\)
increases}

\end{figure}%

\subsubsection{Making Sense of FIM
obtained}\label{making-sense-of-fim-obtained}

\begin{quote}
Still, does our FIM make sense? How can we better understand what FIM is
representing?
\end{quote}

Let's look at the first row of the FIM and reshape it to {[}64, 64{]}.

\begin{figure}

\begin{minipage}{0.33\linewidth}

\includegraphics[width=1\textwidth,height=\textheight]{../../data/N=100/FIM_first_row_multi_100.png}

\subcaption{\label{}FIM{[}0,:{]}}
\end{minipage}%
%
\begin{minipage}{0.33\linewidth}

\includegraphics[width=1\textwidth,height=\textheight]{../../data/N=100/FIM_sec_row_multi_100.png}

\subcaption{\label{}FIM{[}1,:{]}}
\end{minipage}%
%
\begin{minipage}{0.33\linewidth}

\includegraphics[width=1\textwidth,height=\textheight]{../../data/N=100/FIM_third_row_multi_100.png}

\subcaption{\label{}FIM{[}2,:{]}}
\end{minipage}%

\caption{\label{fig-fimrow}Fist, Second, and Third row in FIM}

\end{figure}%

\begin{itemize}
\tightlist
\item
  Like we expected from the definition of FIM, we observe each plot is
  just different linear transformation of
  \(\nabla log p(\{S^t\}^8_{t=1}|K)\)
\item
  As we will see from below, each rows in FIM is noisy version of its
  eigenvector.
\end{itemize}

\subsubsection{\texorpdfstring{How does eigenvectors of FIM look like as
\(M\)
increases?}{How does eigenvectors of FIM look like as M increases?}}\label{how-does-eigenvectors-of-fim-look-like-as-m-increases}

\paragraph{\texorpdfstring{\(M = 1\) (Single
Observation)}{M = 1 (Single Observation)}}\label{m-1-single-observation}

\begin{figure}

\begin{minipage}{0.33\linewidth}

\includegraphics[width=1\textwidth,height=\textheight]{../../data/N=1/FIM_1_first_eig.png}

\subcaption{\label{}First Eigenvector}
\end{minipage}%
%
\begin{minipage}{0.33\linewidth}

\includegraphics[width=1\textwidth,height=\textheight]{../../data/N=1/FIM_1_sec_eig.png}

\subcaption{\label{}Second Eigenvector}
\end{minipage}%
%
\begin{minipage}{0.33\linewidth}

\includegraphics[width=1\textwidth,height=\textheight]{../../data/N=1/FIM_1_third_eig.png}

\subcaption{\label{}Third Eigenvector}
\end{minipage}%

\caption{\label{fig-eig}First three largest eigenvector of FIM}

\end{figure}%

\begin{itemize}
\tightlist
\item
  Even when FIM is computed with single observation, we see that the
  largest eigenvector has the most definition in the shape of
  permeability. Rest of eigenvector looks more like noise.
\end{itemize}

\paragraph{\texorpdfstring{\(M = 10\)}{M = 10}}\label{m-10}

\begin{figure}

\begin{minipage}{0.33\linewidth}

\includegraphics[width=1\textwidth,height=\textheight]{../../data/N=10/FIM_10_first_eig.png}

\subcaption{\label{}First Eigenvector}
\end{minipage}%
%
\begin{minipage}{0.33\linewidth}

\includegraphics[width=1\textwidth,height=\textheight]{../../data/N=10/FIM_10_sec_eig.png}

\subcaption{\label{}Second Eigenvector}
\end{minipage}%
%
\begin{minipage}{0.33\linewidth}

\includegraphics[width=1\textwidth,height=\textheight]{../../data/N=10/FIM_10_third_eig.png}

\subcaption{\label{}Third Eigenvector}
\end{minipage}%

\caption{\label{fig-eig10}First three largest eigenvector of FIM}

\end{figure}%

\paragraph{\texorpdfstring{\(M = 100\)}{M = 100}}\label{m-100}

\begin{figure}

\begin{minipage}{0.33\linewidth}

\includegraphics[width=1\textwidth,height=\textheight]{../../data/N=100/FIM_first_eig.png}

\subcaption{\label{}First Eigenvector}
\end{minipage}%
%
\begin{minipage}{0.33\linewidth}

\includegraphics[width=1\textwidth,height=\textheight]{../../data/N=100/FIM_sec_eig.png}

\subcaption{\label{}Second Eigenvector}
\end{minipage}%
%
\begin{minipage}{0.33\linewidth}

\includegraphics[width=1\textwidth,height=\textheight]{../../data/N=100/FIM_third_eig.png}

\subcaption{\label{}Third Eigenvector}
\end{minipage}%

\caption{\label{fig-eig100}First three largest eigenvector of FIM}

\end{figure}%

\paragraph{\texorpdfstring{\(M = 1000\)}{M = 1000}}\label{m-1000}

\begin{figure}

\begin{minipage}{0.33\linewidth}

\includegraphics[width=1\textwidth,height=\textheight]{../../data/N=1000/FIM_1000_first_eig.png}

\subcaption{\label{}First Eigenvector}
\end{minipage}%
%
\begin{minipage}{0.33\linewidth}

\includegraphics[width=1\textwidth,height=\textheight]{../../data/N=1000/FIM_1000_sec_eig.png}

\subcaption{\label{}Second Eigenvector}
\end{minipage}%
%
\begin{minipage}{0.33\linewidth}

\includegraphics[width=1\textwidth,height=\textheight]{../../data/N=1000/FIM_1000_third_eig.png}

\subcaption{\label{}Third Eigenvector}
\end{minipage}%

\caption{\label{fig-eig1000}First three largest eigenvector of FIM}

\end{figure}%

\begin{itemize}
\tightlist
\item
  As \(M\) increases, we observe flow through the channel clearer.
\item
  We see the boundary of permeability gets clearer.
\item
  In general, it gets less noisy.
\end{itemize}

\subsubsection{\texorpdfstring{How does vector Jacobian product look
like as \(M\)
increases?}{How does vector Jacobian product look like as M increases?}}\label{how-does-vector-jacobian-product-look-like-as-m-increases}

\begin{figure}

\begin{minipage}{0.50\linewidth}

\includegraphics[width=1\textwidth,height=\textheight]{../../data/N=1/FIM_1_vjp.png}

\subcaption{\label{}vjp (\(M=1\))}
\end{minipage}%
%
\begin{minipage}{0.50\linewidth}

\includegraphics[width=1\textwidth,height=\textheight]{../../data/N=10/FIM_10_vjp.png}

\subcaption{\label{}vjp (\(M=10\))}
\end{minipage}%
\newline
\begin{minipage}{0.50\linewidth}

\includegraphics[width=1\textwidth,height=\textheight]{../../data/N=100/FIM_100_vjp.png}

\subcaption{\label{}vjp (\(M=100\))}
\end{minipage}%
%
\begin{minipage}{0.50\linewidth}

\includegraphics[width=1\textwidth,height=\textheight]{../../data/N=1000/FIM_1000_vjp.png}

\subcaption{\label{}vjp (\(M=1000\))}
\end{minipage}%

\caption{\label{fig-eig1000}Normalized Vector Jacobian Product when
vector is the largest eigenvector}

\end{figure}%

\begin{itemize}
\tightlist
\item
  We observe that vector Jacobian product looks more like saturation
  rather than permeability.
\item
  As \(M\) increases, scale in color bar also increases.
\item
  One possible conclusion:

  \begin{itemize}
  \tightlist
  \item
    vjp tells us the location in the spatial distribution (likelihood
    space) where there exists the largest variation, thus have the most
    information on parameter.
  \item
    \(J^Tv\), when \(v\) is the largest eigenvector of FIM, is
    projecting Jacobian onto direction of maximum sensitivity.
  \end{itemize}
\end{itemize}

\subsection{\texorpdfstring{When Random Variable of FIM, \(Y\), is only
Saturation}{When Random Variable of FIM, Y, is only Saturation}}\label{when-random-variable-of-fim-y-is-only-saturation}

After updating the code, we compute FIM of saturation only.

\subsubsection{FIM obtained}\label{fim-obtained}

\begin{itemize}
\tightlist
\item
  We observe that we see off-diagonal structure in this Fisher
  Information Matrix.
\item
  This just means that that are dependency or stronger correlation
  between parameters.
\item
  This might be due to the structure of permeability being heterogenous,
  where point outside the channel does not impact saturation at all.
\end{itemize}

\begin{figure}

\begin{minipage}{0.25\linewidth}

\includegraphics[width=1\textwidth,height=\textheight]{../../data/Saturation_M=1/fim.png}

\subcaption{\label{}\(M = 1\)}
\end{minipage}%
%
\begin{minipage}{0.25\linewidth}

\includegraphics[width=1\textwidth,height=\textheight]{../../data/Saturation_M=10/fim.png}

\subcaption{\label{}\(M = 10\)}
\end{minipage}%
%
\begin{minipage}{0.25\linewidth}

\includegraphics[width=1\textwidth,height=\textheight]{../../data/Saturation_M=100/fim.png}

\subcaption{\label{}\(M = 100\)}
\end{minipage}%
%
\begin{minipage}{0.25\linewidth}

\includegraphics[width=1\textwidth,height=\textheight]{../../data/Saturation_M=100/fim.png}

\subcaption{\label{}\(M = 1000\)}
\end{minipage}%

\caption{\label{fig-fim}FIM{[}:256, :256{]} of different \(M\)}

\end{figure}%

\subsubsection{The Each Rows of FIM}\label{the-each-rows-of-fim}

Each row of FIM can be considered as some linear combination of
gradient. Each row represents each grid point of permeability that is
perturbed, and the plot we are seeing shows how likelihood changes when
the certain grid point of permeability is perturbed.

When \(M=1\),

\begin{figure}

\begin{minipage}{0.33\linewidth}

\includegraphics[width=1\textwidth,height=\textheight]{../../data/Saturation_M=1/fim_1strow.png}

\subcaption{\label{}\(i = 1\)}
\end{minipage}%
%
\begin{minipage}{0.33\linewidth}

\includegraphics[width=1\textwidth,height=\textheight]{../../data/Saturation_M=1/fim_500throw.png}

\subcaption{\label{}\(i = 500\)}
\end{minipage}%
%
\begin{minipage}{0.33\linewidth}

\includegraphics[width=1\textwidth,height=\textheight]{../../data/Saturation_M=1/fim_2000throw.png}

\subcaption{\label{}\(i = 2000\)}
\end{minipage}%

\caption{\label{fig-eig1000}FIM of each rows when \(M=1\)}

\end{figure}%

When \(M=10\),

\begin{figure}

\begin{minipage}{0.33\linewidth}

\includegraphics[width=1\textwidth,height=\textheight]{../../data/Saturation_M=10/fim_1strow.png}

\subcaption{\label{}\(i = 1\)}
\end{minipage}%
%
\begin{minipage}{0.33\linewidth}

\includegraphics[width=1\textwidth,height=\textheight]{../../data/Saturation_M=10/fim_500throw.png}

\subcaption{\label{}\(i = 500\)}
\end{minipage}%
%
\begin{minipage}{0.33\linewidth}

\includegraphics[width=1\textwidth,height=\textheight]{../../data/Saturation_M=10/fim_2000throw.png}

\subcaption{\label{}\(i = 2000\)}
\end{minipage}%

\caption{\label{fig-eig1000}FIM of each rows when \(M=10\)}

\end{figure}%

When \(M=100\),

\begin{figure}

\begin{minipage}{0.33\linewidth}

\includegraphics[width=1\textwidth,height=\textheight]{../../data/Saturation_M=100/fim_1strow.png}

\subcaption{\label{}\(i = 1\)}
\end{minipage}%
%
\begin{minipage}{0.33\linewidth}

\includegraphics[width=1\textwidth,height=\textheight]{../../data/Saturation_M=100/fim_500throw.png}

\subcaption{\label{}\(i = 500\)}
\end{minipage}%
%
\begin{minipage}{0.33\linewidth}

\includegraphics[width=1\textwidth,height=\textheight]{../../data/Saturation_M=100/fim_2000throw.png}

\subcaption{\label{}\(i = 2000\)}
\end{minipage}%

\caption{\label{fig-eig1000}FIM of each rows when \(M=100\)}

\end{figure}%

When \(M=1000\),

\begin{figure}

\begin{minipage}{0.33\linewidth}

\includegraphics[width=1\textwidth,height=\textheight]{../../data/Saturation_M=1000/fim_1strow.png}

\subcaption{\label{}\(i = 1\)}
\end{minipage}%
%
\begin{minipage}{0.33\linewidth}

\includegraphics[width=1\textwidth,height=\textheight]{../../data/Saturation_M=1000/fim_500throw.png}

\subcaption{\label{}\(i = 500\)}
\end{minipage}%
%
\begin{minipage}{0.33\linewidth}

\includegraphics[width=1\textwidth,height=\textheight]{../../data/Saturation_M=1000/fim_2000throw.png}

\subcaption{\label{}\(i = 2000\)}
\end{minipage}%

\caption{\label{fig-eig1000}FIM of each rows when \(M=1000\)}

\end{figure}%

\subsubsection{Eigenvector of FIM}\label{eigenvector-of-fim}

\begin{figure}

\begin{minipage}{0.25\linewidth}

\includegraphics[width=1\textwidth,height=\textheight]{../../data/Saturation_M=1/eigvec_1st.png}

\subcaption{\label{}\(M = 1\)}
\end{minipage}%
%
\begin{minipage}{0.25\linewidth}

\includegraphics[width=1\textwidth,height=\textheight]{../../data/Saturation_M=10/eigvec_1st.png}

\subcaption{\label{}\(M = 10\)}
\end{minipage}%
%
\begin{minipage}{0.25\linewidth}

\includegraphics[width=1\textwidth,height=\textheight]{../../data/Saturation_M=100/eigvec_1st.png}

\subcaption{\label{}\(M = 100\)}
\end{minipage}%
%
\begin{minipage}{0.25\linewidth}

\includegraphics[width=1\textwidth,height=\textheight]{../../data/Saturation_M=1000/eigvec_1st.png}

\subcaption{\label{}\(M = 1000\)}
\end{minipage}%

\caption{\label{fig-eig1000}The largest eigenvector of FIM of different
\(M\)}

\end{figure}%

\subsubsection{Vector Jacobian Product
Obtained}\label{vector-jacobian-product-obtained}

\subsubsection{Eigenvector of FIM}\label{eigenvector-of-fim-1}

\begin{figure}

\begin{minipage}{0.25\linewidth}

\includegraphics[width=1\textwidth,height=\textheight]{../../data/Saturation_M=1/vjp.png}

\subcaption{\label{}\(M = 1\)}
\end{minipage}%
%
\begin{minipage}{0.25\linewidth}

\includegraphics[width=1\textwidth,height=\textheight]{../../data/Saturation_M=10/vjp.png}

\subcaption{\label{}\(M = 10\)}
\end{minipage}%
%
\begin{minipage}{0.25\linewidth}

\includegraphics[width=1\textwidth,height=\textheight]{../../data/Saturation_M=100/vjp.png}

\subcaption{\label{}\(M = 100\)}
\end{minipage}%
%
\begin{minipage}{0.25\linewidth}

\includegraphics[width=1\textwidth,height=\textheight]{../../data/Saturation_M=1000/vjp.png}

\subcaption{\label{}\(M = 1000\)}
\end{minipage}%

\caption{\label{fig-eig1000}The largest eigenvector of FIM of different
\(M\)}

\end{figure}%

\section{Training Result}\label{training-result}

We first training with following configuration:

\begin{itemize}
\tightlist
\item
  Training , Test = {[}1800, 200{]}
\item
  Batch size = 100
\item
  Number of Epoch = 1000
\end{itemize}

\begin{longtable}[]{@{}
  >{\raggedright\arraybackslash}p{(\columnwidth - 4\tabcolsep) * \real{0.2239}}
  >{\raggedright\arraybackslash}p{(\columnwidth - 4\tabcolsep) * \real{0.3881}}
  >{\raggedright\arraybackslash}p{(\columnwidth - 4\tabcolsep) * \real{0.3881}}@{}}
\caption{Loss Table}\tabularnewline
\toprule\noalign{}
\begin{minipage}[b]{\linewidth}\raggedright
\end{minipage} & \begin{minipage}[b]{\linewidth}\raggedright
Train Loss
\end{minipage} & \begin{minipage}[b]{\linewidth}\raggedright
Test Loss
\end{minipage} \\
\midrule\noalign{}
\endfirsthead
\toprule\noalign{}
\begin{minipage}[b]{\linewidth}\raggedright
\end{minipage} & \begin{minipage}[b]{\linewidth}\raggedright
Train Loss
\end{minipage} & \begin{minipage}[b]{\linewidth}\raggedright
Test Loss
\end{minipage} \\
\midrule\noalign{}
\endhead
\bottomrule\noalign{}
\endlastfoot
& MSE/GM & MSE \\
\(FNO_{MSE}\) & \(3.3622 \times 10^{-8}\) & \(8.4016 \times 10^{-8}\) \\
\(FNO_{GM}\) & \(2.6428 \times 10^{-7}\) & \(1.5976 \times 10^{-7}\) \\
\end{longtable}

\begin{figure}

\begin{minipage}{0.50\linewidth}

\includegraphics[width=1\textwidth,height=\textheight]{../../test/all_loss.png}

\subcaption{\label{}All loss}
\end{minipage}%
%
\begin{minipage}{0.50\linewidth}

\includegraphics[width=1\textwidth,height=\textheight]{../../test/GM_term.png}

\subcaption{\label{}Only GM Term}
\end{minipage}%

\caption{\label{fig-loss}Loss plots}

\end{figure}%

\subsection{MSE}\label{mse}

\subsubsection{Forward Simulation}\label{forward-simulation}

\begin{figure}[H]

{\centering \includegraphics[width=1\textwidth,height=\textheight]{../../plot/GCS_channel_plot/FNO_GCS_lowest_MSE_True.png}

}

\caption{True Saturation}

\end{figure}%%
\begin{figure}[H]

{\centering \includegraphics[width=1\textwidth,height=\textheight]{../../plot/GCS_channel_plot/FNO_GCS_lowest_MSE_Pred.png}

}

\caption{Predicted Saturation}

\end{figure}%%
\begin{figure}[H]

{\centering \includegraphics[width=1\textwidth,height=\textheight]{../../plot/GCS_channel_plot/FNO_GCS_lowest_MSE_diff.png}

}

\caption{Absolute Difference}

\end{figure}%

\subsubsection{Learned and True vjp when just trained with
MSE}\label{learned-and-true-vjp-when-just-trained-with-mse}

We observe

\begin{enumerate}
\def\labelenumi{\arabic{enumi}.}
\tightlist
\item
  Scale in the color bar does not match.
\item
  The learned vjp looks noisy as there are some colors showing in the
  part where it should be just white.
\end{enumerate}

\begin{figure}[H]

{\centering \includegraphics[width=1\textwidth,height=\textheight]{../../plot/GCS_channel_plot/training/MSE/true_vjp_1.png}

}

\caption{True vjp}

\end{figure}%%
\begin{figure}[H]

{\centering \includegraphics[width=1\textwidth,height=\textheight]{../../plot/GCS_channel_plot/training/MSE/learned_vjp_990.png}

}

\caption{Learned vjp}

\end{figure}%%
\begin{figure}[H]

{\centering \includegraphics[width=1\textwidth,height=\textheight]{../../plot/GCS_channel_plot/training/MSE/diff_vjp_990.png}

}

\caption{Absolute Difference}

\end{figure}%

\subsection{Gradient-Matching}\label{gradient-matching}

\subsubsection{Forward Simulation}\label{forward-simulation-1}

\begin{figure}[H]

{\centering \includegraphics[width=1\textwidth,height=\textheight]{../../plot/GCS_channel_plot/FNO_GCS_lowest_JAC_True.png}

}

\caption{True Saturation}

\end{figure}%%
\begin{figure}[H]

{\centering \includegraphics[width=1\textwidth,height=\textheight]{../../plot/GCS_channel_plot/FNO_GCS_lowest_JAC_Pred.png}

}

\caption{Predicted Saturation}

\end{figure}%%
\begin{figure}[H]

{\centering \includegraphics[width=1\textwidth,height=\textheight]{../../plot/GCS_channel_plot/FNO_GCS_lowest_JAC_diff.png}

}

\caption{Absolute Difference}

\end{figure}%

\subsubsection{Learned and True vjp}\label{learned-and-true-vjp}

We now observe that the learned and the true vjp matches well. Unlike
MSE model, we observe

\begin{enumerate}
\def\labelenumi{\arabic{enumi}.}
\tightlist
\item
  The scale of color bar matches correctly.
\item
  The plot does not look noisy.
\end{enumerate}

\begin{figure}[H]

{\centering \includegraphics[width=1\textwidth,height=\textheight]{../../plot/GCS_channel_plot/training/JAC/true_vjp_1.png}

}

\caption{True vjp}

\end{figure}%%
\begin{figure}[H]

{\centering \includegraphics[width=1\textwidth,height=\textheight]{../../plot/GCS_channel_plot/training/JAC/learned_vjp_990.png}

}

\caption{Learned vjp}

\end{figure}%%
\begin{figure}[H]

{\centering \includegraphics[width=1\textwidth,height=\textheight]{../../plot/GCS_channel_plot/training/JAC/diff_vjp_990.png}

}

\caption{Absolute Difference}

\end{figure}%

\subsection{Future Step}\label{future-step}

\begin{enumerate}
\def\labelenumi{\arabic{enumi}.}
\tightlist
\item
  TODO: Debug NS eigenvector and vjp.
\item
  TODO: Want to generate the full dataset for Francis' dataset (which
  might take 1 or 2 days).
\item
  TODO: Try it on Jason's dataset (Now that we fixed the problem with
  FIM computation, we are optimistic about the experiment, so we want to
  try it again.)
\end{enumerate}

\subsection{Question}\label{question}

\begin{enumerate}
\def\labelenumi{\arabic{enumi}.}
\tightlist
\item
  Do we want to train both models for a longer time?
\end{enumerate}




\end{document}
