% Options for packages loaded elsewhere
\PassOptionsToPackage{unicode}{hyperref}
\PassOptionsToPackage{hyphens}{url}
\PassOptionsToPackage{dvipsnames,svgnames,x11names}{xcolor}
%
\documentclass[
]{article}

\usepackage{amsmath,amssymb}
\usepackage{iftex}
\ifPDFTeX
  \usepackage[T1]{fontenc}
  \usepackage[utf8]{inputenc}
  \usepackage{textcomp} % provide euro and other symbols
\else % if luatex or xetex
  \usepackage{unicode-math}
  \defaultfontfeatures{Scale=MatchLowercase}
  \defaultfontfeatures[\rmfamily]{Ligatures=TeX,Scale=1}
\fi
\usepackage{lmodern}
\ifPDFTeX\else  
    % xetex/luatex font selection
    \setmainfont[]{Latin Modern Roman}
  \setmathfont[]{Latin Modern Math}
\fi
% Use upquote if available, for straight quotes in verbatim environments
\IfFileExists{upquote.sty}{\usepackage{upquote}}{}
\IfFileExists{microtype.sty}{% use microtype if available
  \usepackage[]{microtype}
  \UseMicrotypeSet[protrusion]{basicmath} % disable protrusion for tt fonts
}{}
\makeatletter
\@ifundefined{KOMAClassName}{% if non-KOMA class
  \IfFileExists{parskip.sty}{%
    \usepackage{parskip}
  }{% else
    \setlength{\parindent}{0pt}
    \setlength{\parskip}{6pt plus 2pt minus 1pt}}
}{% if KOMA class
  \KOMAoptions{parskip=half}}
\makeatother
\usepackage{xcolor}
\setlength{\emergencystretch}{3em} % prevent overfull lines
\setcounter{secnumdepth}{5}
% Make \paragraph and \subparagraph free-standing
\makeatletter
\ifx\paragraph\undefined\else
  \let\oldparagraph\paragraph
  \renewcommand{\paragraph}{
    \@ifstar
      \xxxParagraphStar
      \xxxParagraphNoStar
  }
  \newcommand{\xxxParagraphStar}[1]{\oldparagraph*{#1}\mbox{}}
  \newcommand{\xxxParagraphNoStar}[1]{\oldparagraph{#1}\mbox{}}
\fi
\ifx\subparagraph\undefined\else
  \let\oldsubparagraph\subparagraph
  \renewcommand{\subparagraph}{
    \@ifstar
      \xxxSubParagraphStar
      \xxxSubParagraphNoStar
  }
  \newcommand{\xxxSubParagraphStar}[1]{\oldsubparagraph*{#1}\mbox{}}
  \newcommand{\xxxSubParagraphNoStar}[1]{\oldsubparagraph{#1}\mbox{}}
\fi
\makeatother


\providecommand{\tightlist}{%
  \setlength{\itemsep}{0pt}\setlength{\parskip}{0pt}}\usepackage{longtable,booktabs,array}
\usepackage{calc} % for calculating minipage widths
% Correct order of tables after \paragraph or \subparagraph
\usepackage{etoolbox}
\makeatletter
\patchcmd\longtable{\par}{\if@noskipsec\mbox{}\fi\par}{}{}
\makeatother
% Allow footnotes in longtable head/foot
\IfFileExists{footnotehyper.sty}{\usepackage{footnotehyper}}{\usepackage{footnote}}
\makesavenoteenv{longtable}
\usepackage{graphicx}
\makeatletter
\def\maxwidth{\ifdim\Gin@nat@width>\linewidth\linewidth\else\Gin@nat@width\fi}
\def\maxheight{\ifdim\Gin@nat@height>\textheight\textheight\else\Gin@nat@height\fi}
\makeatother
% Scale images if necessary, so that they will not overflow the page
% margins by default, and it is still possible to overwrite the defaults
% using explicit options in \includegraphics[width, height, ...]{}
\setkeys{Gin}{width=\maxwidth,height=\maxheight,keepaspectratio}
% Set default figure placement to htbp
\makeatletter
\def\fps@figure{htbp}
\makeatother

\usepackage{arxiv}
\usepackage{orcidlink}
\usepackage{amsmath}
\usepackage[T1]{fontenc}
\makeatletter
\@ifpackageloaded{caption}{}{\usepackage{caption}}
\AtBeginDocument{%
\ifdefined\contentsname
  \renewcommand*\contentsname{Table of contents}
\else
  \newcommand\contentsname{Table of contents}
\fi
\ifdefined\listfigurename
  \renewcommand*\listfigurename{List of Figures}
\else
  \newcommand\listfigurename{List of Figures}
\fi
\ifdefined\listtablename
  \renewcommand*\listtablename{List of Tables}
\else
  \newcommand\listtablename{List of Tables}
\fi
\ifdefined\figurename
  \renewcommand*\figurename{Figure}
\else
  \newcommand\figurename{Figure}
\fi
\ifdefined\tablename
  \renewcommand*\tablename{Table}
\else
  \newcommand\tablename{Table}
\fi
}
\@ifpackageloaded{float}{}{\usepackage{float}}
\floatstyle{ruled}
\@ifundefined{c@chapter}{\newfloat{codelisting}{h}{lop}}{\newfloat{codelisting}{h}{lop}[chapter]}
\floatname{codelisting}{Listing}
\newcommand*\listoflistings{\listof{codelisting}{List of Listings}}
\makeatother
\makeatletter
\makeatother
\makeatletter
\@ifpackageloaded{caption}{}{\usepackage{caption}}
\@ifpackageloaded{subcaption}{}{\usepackage{subcaption}}
\makeatother
\makeatletter
\@ifpackageloaded{sidenotes}{}{\usepackage{sidenotes}}
\@ifpackageloaded{marginnote}{}{\usepackage{marginnote}}
\makeatother

\ifLuaTeX
  \usepackage{selnolig}  % disable illegal ligatures
\fi
\usepackage{bookmark}

\IfFileExists{xurl.sty}{\usepackage{xurl}}{} % add URL line breaks if available
\urlstyle{same} % disable monospaced font for URLs
\hypersetup{
  pdftitle={Data Generation},
  pdfauthor={Jayjay, Tuna, Jason, Richard},
  colorlinks=true,
  linkcolor={blue},
  filecolor={Maroon},
  citecolor={Blue},
  urlcolor={Blue},
  pdfcreator={LaTeX via pandoc}}


\renewcommand{\today}{2024-09-30}
\newcommand{\runninghead}{A Preprint }
\renewcommand{\runninghead}{A Preprint }
\title{Data Generation}
\def\asep{\\\\\\ } % default: all authors on same column
\author{\textbf{Jayjay, Tuna, Jason, Richard}\\}
\date{2024-09-30}
\begin{document}
\maketitle


\section{Surrogate Modeling for Which
System?}\label{surrogate-modeling-for-which-system}

\begin{enumerate}
\def\labelenumi{\arabic{enumi}.}
\tightlist
\item
  Simplified Geological Carbon Storage (Francis' paper)
\item
  Incompressible Navier Stokes
\end{enumerate}

\section{Twophase flow for the CO2
saturation}\label{twophase-flow-for-the-co2-saturation}

\begin{itemize}
\tightlist
\item
  We regenerate Francis' dataset, and additionally compute Fisher
  Information Matrix as well.
\item
  For the purpose of validation, we currently form full Fisher
  Infromation Matrix and then compute eigenvector.
\item
  Our next step will be low rank approximation or trace estimation so
  that we don't have to form the full matrix.
\end{itemize}

\subsection{Dataset}\label{dataset}

Our dataset consists of \(2000\) pairs of \(\{K, S^t(K)\}_{t=1}^8\).

\begin{figure}

\begin{minipage}{0.50\linewidth}

\centering{

\includegraphics[width=0.8\textwidth,height=\textheight]{../../data/Ks_0.png}

}

\subcaption{\label{fig-surus}K0}

\end{minipage}%
%
\begin{minipage}{0.50\linewidth}

\centering{

\includegraphics[width=0.8\textwidth,height=\textheight]{../../data/Ks_1.png}

}

\subcaption{\label{fig-hanno}K1}

\end{minipage}%

\caption{\label{fig-K}Example Permeability Model}

\end{figure}%

\begin{figure}

\begin{minipage}{\linewidth}

\centering{

\includegraphics[width=1\textwidth,height=\textheight]{../../data/Snew_series.png}

}

\subcaption{\label{fig-S0}Time Series of Saturation of K0}

\end{minipage}%
\newline
\begin{minipage}{\linewidth}

\centering{

\includegraphics[width=1\textwidth,height=\textheight]{../../data/Snew_series1.png}

}

\subcaption{\label{fig-S1}Time Series of Saturation of K1}

\end{minipage}%

\caption{\label{fig-S}Example Saturation Time Series}

\end{figure}%

\subsection{Fisher Information Matrix}\label{fisher-information-matrix}

\begin{itemize}
\tightlist
\item
  To find the optimal number of observations, \(M\), we visualize
  eigenvector and vector jacobian product.
\item
  We observe that as \(M\) increases, the clearer we see the boundary of
  the permeabiltiy, which will be more informative during training and
  inference. \sidenote{\footnotesize \href{https://www.overleaf.com/1149716711hxnvfbyfpzvb\#a799ce}{Note
    on Learning Problem}.}
\item
  Given 1 pair of dataset, \(\{K, S^t(K)\}^8_{t=1}\), we get a single
  FIM.
\end{itemize}

\subsubsection{Computing Fisher Information Matrix for each
datapoint}\label{computing-fisher-information-matrix-for-each-datapoint}

We consider a realistic scenario when we only have access to samples,
but not distribution. When \(N\) is number of samples and
\(X \in \mathbb{R}^{d \times d}\), neural network model \(F_{nn}\)
learns mapping from \(X_i \rightarrow Y_i\). For each pair of
\(\left\{X_i, Y_i \right\}^N_{i=1}\), we generate
\(\left\{FIM_i\right\}_{i=1}^{N}\).

\begin{itemize}
\tightlist
\item
  \(N\) : number of data points, \(\left\{X_i, Y_i \right\}\)
\item
  \(M\) : number of observation, \(Y\)
\end{itemize}

\begin{quote}
\[ \left\{ X_i \right\}^N_{i=1} \sim p_X(X), \: \epsilon \sim \mathcal{N}(0, \Sigma), \: \Sigma = I
\] For a single data pair, we generate multiple observations.
\[Y_{i, J} = F(X_i) + \epsilon_{i, J}, \quad where \left\{ \epsilon_{i,J}\right\}^{N,M}_{i,J= 1,1}\]
As we assumed Gaussian, we define likelihood as following.
\[p(Y_{i,J}|X_i) = e^{-\frac{1}{2}\|Y_{i,J}-F(X_i)\|^2_2}\]
\[log \: p(Y_{i,J}|X_i) \approx \frac{1}{\Sigma}\|Y_{i,J}-F(X_i)\|^2_2\]
A FIM for a single data pair \(i\) is:
\[FIM_i = \mathbb{E}_{Y_{i, \{J\}^m_{i=1}} \sim p(Y_{i,J}|X_i)} \left[ \left(\nabla log \: p(Y_{i,J}|X_i)\right)\left(\nabla log \: p(Y_{i,J}|X_i)\right)^T\right]\]
\end{quote}

\subsubsection{How does FIM change as number of observation
increases?}\label{how-does-fim-change-as-number-of-observation-increases}

FIM is expectation of covariance of derivative of log likelihood. As we
expected, we see clearer definition in diagonal relationship as \(M\)
increases.

\begin{figure}

\begin{minipage}{0.33\linewidth}

\includegraphics[width=1\textwidth,height=\textheight]{../../data/FIM/FIM0_sub0.png}

\subcaption{\label{}M = 1}
\end{minipage}%
%
\begin{minipage}{0.33\linewidth}

\includegraphics[width=1\textwidth,height=\textheight]{../../data/FIM/FIM0_sub0_multi_10.png}

\subcaption{\label{}M = 10}
\end{minipage}%
%
\begin{minipage}{0.33\linewidth}

\includegraphics[width=1\textwidth,height=\textheight]{../../data/FIM/FIM0_sub0_multi_100.png}

\subcaption{\label{}M = 100}
\end{minipage}%

\caption{\label{fig-fim}Change in FIM{[}:256, :256{]} of single data
pair \(\{K, S^t(K)\}^8_{t=1}\) as number of observation, \(M\)
increases}

\end{figure}%

\subsubsection{Making Sense of FIM
obtained}\label{making-sense-of-fim-obtained}

\begin{quote}
Still, does our FIM make sense? How can we better understand what FIM is
representing?
\end{quote}

Let's look at the first row of the FIM and reshape it to {[}64, 64{]}.

\begin{figure}

\begin{minipage}{0.33\linewidth}

\includegraphics[width=1\textwidth,height=\textheight]{../../data/N=100/FIM_first_row_multi_100.png}

\subcaption{\label{}FIM{[}0,:{]}}
\end{minipage}%
%
\begin{minipage}{0.33\linewidth}

\includegraphics[width=1\textwidth,height=\textheight]{../../data/N=100/FIM_sec_row_multi_100.png}

\subcaption{\label{}FIM{[}1,:{]}}
\end{minipage}%
%
\begin{minipage}{0.33\linewidth}

\includegraphics[width=1\textwidth,height=\textheight]{../../data/N=100/FIM_third_row_multi_100.png}

\subcaption{\label{}FIM{[}2,:{]}}
\end{minipage}%

\caption{\label{fig-fimrow}Fist, Second, and Third row in FIM}

\end{figure}%

\begin{itemize}
\tightlist
\item
  Like we expected from the definition of FIM, we observe each plot is
  just different linear transformation of
  \(\nabla log p(\{S^t\}^8_{t=1}|K)\)
\item
  As we will see from below, each rows in FIM is noisy version of its
  eigenvector.
\end{itemize}

\subsubsection{\texorpdfstring{How does eigenvectors of FIM look like as
\(M\)
increases?}{How does eigenvectors of FIM look like as M increases?}}\label{how-does-eigenvectors-of-fim-look-like-as-m-increases}

\paragraph{\texorpdfstring{\(M = 1\) (Single
Observation)}{M = 1 (Single Observation)}}\label{m-1-single-observation}

\begin{figure}

\begin{minipage}{0.33\linewidth}

\includegraphics[width=1\textwidth,height=\textheight]{../../data/N=1/FIM_1_first_eig.png}

\subcaption{\label{}First Eigenvector}
\end{minipage}%
%
\begin{minipage}{0.33\linewidth}

\includegraphics[width=1\textwidth,height=\textheight]{../../data/N=1/FIM_1_sec_eig.png}

\subcaption{\label{}Second Eigenvector}
\end{minipage}%
%
\begin{minipage}{0.33\linewidth}

\includegraphics[width=1\textwidth,height=\textheight]{../../data/N=1/FIM_1_third_eig.png}

\subcaption{\label{}Third Eigenvector}
\end{minipage}%

\caption{\label{fig-eig}First three largest eigenvector of FIM}

\end{figure}%

\begin{itemize}
\tightlist
\item
  Even when FIM is computed with single observation, we see that the
  largest eigenvector has the most definition in the shape of
  permeability. Rest of eigenvector looks more like noise.
\end{itemize}

\paragraph{\texorpdfstring{\(M = 10\)}{M = 10}}\label{m-10}

\begin{figure}

\begin{minipage}{0.33\linewidth}

\includegraphics[width=1\textwidth,height=\textheight]{../../data/N=10/FIM_10_first_eig.png}

\subcaption{\label{}First Eigenvector}
\end{minipage}%
%
\begin{minipage}{0.33\linewidth}

\includegraphics[width=1\textwidth,height=\textheight]{../../data/N=10/FIM_10_sec_eig.png}

\subcaption{\label{}Second Eigenvector}
\end{minipage}%
%
\begin{minipage}{0.33\linewidth}

\includegraphics[width=1\textwidth,height=\textheight]{../../data/N=10/FIM_10_third_eig.png}

\subcaption{\label{}Third Eigenvector}
\end{minipage}%

\caption{\label{fig-eig10}First three largest eigenvector of FIM}

\end{figure}%

\paragraph{\texorpdfstring{\(M = 100\)}{M = 100}}\label{m-100}

\begin{figure}

\begin{minipage}{0.33\linewidth}

\includegraphics[width=1\textwidth,height=\textheight]{../../data/N=100/FIM_first_eig.png}

\subcaption{\label{}First Eigenvector}
\end{minipage}%
%
\begin{minipage}{0.33\linewidth}

\includegraphics[width=1\textwidth,height=\textheight]{../../data/N=100/FIM_sec_eig.png}

\subcaption{\label{}Second Eigenvector}
\end{minipage}%
%
\begin{minipage}{0.33\linewidth}

\includegraphics[width=1\textwidth,height=\textheight]{../../data/N=100/FIM_third_eig.png}

\subcaption{\label{}Third Eigenvector}
\end{minipage}%

\caption{\label{fig-eig100}First three largest eigenvector of FIM}

\end{figure}%

\paragraph{\texorpdfstring{\(M = 1000\)}{M = 1000}}\label{m-1000}

\begin{figure}

\begin{minipage}{0.33\linewidth}

\includegraphics[width=1\textwidth,height=\textheight]{../../data/N=1000/FIM_1000_first_eig.png}

\subcaption{\label{}First Eigenvector}
\end{minipage}%
%
\begin{minipage}{0.33\linewidth}

\includegraphics[width=1\textwidth,height=\textheight]{../../data/N=1000/FIM_1000_sec_eig.png}

\subcaption{\label{}Second Eigenvector}
\end{minipage}%
%
\begin{minipage}{0.33\linewidth}

\includegraphics[width=1\textwidth,height=\textheight]{../../data/N=1000/FIM_1000_third_eig.png}

\subcaption{\label{}Third Eigenvector}
\end{minipage}%

\caption{\label{fig-eig1000}First three largest eigenvector of FIM}

\end{figure}%

\begin{itemize}
\tightlist
\item
  As \(M\) increases, we observe flow through the channel clearer.
\item
  We see the boundary of permeability gets clearer.
\item
  In general, it gets less noisy.
\end{itemize}

\subsubsection{\texorpdfstring{How does vector Jacobian product look
like as \(M\)
increases?}{How does vector Jacobian product look like as M increases?}}\label{how-does-vector-jacobian-product-look-like-as-m-increases}

\begin{figure}

\begin{minipage}{0.50\linewidth}

\includegraphics[width=1\textwidth,height=\textheight]{../../data/N=1/FIM_1_vjp.png}

\subcaption{\label{}vjp (\(M=1\))}
\end{minipage}%
%
\begin{minipage}{0.50\linewidth}

\includegraphics[width=1\textwidth,height=\textheight]{../../data/N=10/FIM_10_vjp.png}

\subcaption{\label{}vjp (\(M=10\))}
\end{minipage}%
\newline
\begin{minipage}{0.50\linewidth}

\includegraphics[width=1\textwidth,height=\textheight]{../../data/N=100/FIM_100_vjp.png}

\subcaption{\label{}vjp (\(M=100\))}
\end{minipage}%
%
\begin{minipage}{0.50\linewidth}

\includegraphics[width=1\textwidth,height=\textheight]{../../data/N=1000/FIM_1000_vjp.png}

\subcaption{\label{}vjp (\(M=1000\))}
\end{minipage}%

\caption{\label{fig-eig1000}Normalized Vector Jacobian Product when
vector is the largest eigenvector}

\end{figure}%

\begin{itemize}
\tightlist
\item
  We observe that vector Jacobian product looks more like saturation
  rather than permeability.
\item
  As \(M\) increases, scale in color bar also increases.
\item
  One possible conclusion:

  \begin{itemize}
  \tightlist
  \item
    vjp tells us the location in the spatial distribution (likelihood
    space) where there exists the largest variation, thus have the most
    information on parameter.
  \item
    \(J^Tv\), when \(v\) is the largest eigenvector of FIM, is
    projecting Jacobian onto direction of maximum sensitivity.
  \end{itemize}
\end{itemize}

\section{Incompressible Navier
Stokes}\label{incompressible-navier-stokes}

\subsection{Dataset}\label{dataset-1}

\begin{figure}

\begin{minipage}{0.50\linewidth}

\includegraphics[width=1\textwidth,height=\textheight]{../../plot/NS_plot/input.png}

\subcaption{\label{}Vorticity at \(t=0\)}
\end{minipage}%
%
\begin{minipage}{0.50\linewidth}

\includegraphics[width=1\textwidth,height=\textheight]{../../plot/NS_plot/output.png}

\subcaption{\label{}Vorticity at \(t=40\)}
\end{minipage}%

\caption{\label{fig-vort}The first and the last vorticity in a single
time series}

\end{figure}%

Our dataset consists of 50 pairs of
\(\{\varphi^{t-1}(x_0), \varphi^t(x_0)\}^T_{t=1}\), where \(T=44\).
Initial vorticities are a Gaussian Random Fields.

\subsection{Fisher Information
Matrix}\label{fisher-information-matrix-1}

\subsubsection{How do we compute FIM?}\label{how-do-we-compute-fim}

\(FIM = \left(\nabla log p( \varphi^t(x_0) | \varphi^0(x_0))\right)\left(\nabla log p( \varphi^t(x_0) | \varphi^0(x_0))\right)^T\)

\begin{itemize}
\tightlist
\item
  Just means that we are computing FIM with respect to the initial
  vorticity, \(\varphi^t(x_0)\).
\end{itemize}

\subsubsection{\texorpdfstring{How does FIM looks like as \(M\)
changes?}{How does FIM looks like as M changes?}}\label{how-does-fim-looks-like-as-m-changes}

\begin{figure}

\begin{minipage}{0.50\linewidth}

\includegraphics[width=1\textwidth,height=\textheight]{../../plot/NS_plot/10/fim_sub_0_9_t=0.png}

\subcaption{\label{}\(M=10\)}
\end{minipage}%
%
\begin{minipage}{0.50\linewidth}

\includegraphics[width=1\textwidth,height=\textheight]{../../plot/NS_plot/100/fim_sub_0_9_t=0.png}

\subcaption{\label{}\(M=100\)}
\end{minipage}%

\caption{\label{fig-fim_NS}FIM{[}:100, :100{]} of varying \(M\)}

\end{figure}%

\subsubsection{Making Sense of FIM
obtained}\label{making-sense-of-fim-obtained-1}

\begin{quote}
Still, does our FIM make sense? How can we better understand what FIM is
representing?
\end{quote}

Let's look at the first row of the Fisher Information Matrix and reshape
it to {[}64,64{]}.

\begin{figure}

\begin{minipage}{0.50\linewidth}

\includegraphics[width=1\textwidth,height=\textheight]{../../plot/NS_plot/FIM/past/fim_sub_reshape_0.png}

\subcaption{\label{}FIM{[}0, :{]}}
\end{minipage}%
%
\begin{minipage}{0.50\linewidth}

\includegraphics[width=1\textwidth,height=\textheight]{../../plot/NS_plot/input.png}

\subcaption{\label{}Input Vorticity}
\end{minipage}%

\caption{\label{fig-eig100}Comparison of the input parameter with the
first element of FIM}

\end{figure}%

Also, let's look at how the first row of the FIM changes as time
evolves. When \(M=10\),

\begin{figure}

\begin{minipage}{0.20\linewidth}

\includegraphics[width=1\textwidth,height=\textheight]{../../plot/NS_plot/10/fim_sub_reshape_0_9_t=0.png}

\subcaption{\label{}\(t=1\)}
\end{minipage}%
%
\begin{minipage}{0.20\linewidth}

\includegraphics[width=1\textwidth,height=\textheight]{../../plot/NS_plot/10/fim_sub_reshape_0_9_t=4.png}

\subcaption{\label{}\(t=5\)}
\end{minipage}%
%
\begin{minipage}{0.20\linewidth}

\includegraphics[width=1\textwidth,height=\textheight]{../../plot/NS_plot/10/fim_sub_reshape_0_9_t=9.png}

\subcaption{\label{}\(t=10\)}
\end{minipage}%
%
\begin{minipage}{0.20\linewidth}

\includegraphics[width=1\textwidth,height=\textheight]{../../plot/NS_plot/10/fim_sub_reshape_0_9_t=14.png}

\subcaption{\label{}\(t=15\)}
\end{minipage}%
%
\begin{minipage}{0.20\linewidth}

\includegraphics[width=1\textwidth,height=\textheight]{../../plot/NS_plot/10/fim_sub_reshape_0_9_t=19.png}

\subcaption{\label{}\(t=20\)}
\end{minipage}%
\newline
\begin{minipage}{0.20\linewidth}

\includegraphics[width=1\textwidth,height=\textheight]{../../plot/NS_plot/10/fim_sub_reshape_0_9_t=24.png}

\subcaption{\label{}\(t=25\)}
\end{minipage}%
%
\begin{minipage}{0.20\linewidth}

\includegraphics[width=1\textwidth,height=\textheight]{../../plot/NS_plot/10/fim_sub_reshape_0_9_t=29.png}

\subcaption{\label{}\(t=30\)}
\end{minipage}%
%
\begin{minipage}{0.20\linewidth}

\includegraphics[width=1\textwidth,height=\textheight]{../../plot/NS_plot/10/fim_sub_reshape_0_9_t=34.png}

\subcaption{\label{}\(t=35\)}
\end{minipage}%
%
\begin{minipage}{0.20\linewidth}

\includegraphics[width=1\textwidth,height=\textheight]{../../plot/NS_plot/10/fim_sub_reshape_0_9_t=39.png}

\subcaption{\label{}\(t=40\)}
\end{minipage}%
%
\begin{minipage}{0.20\linewidth}

\includegraphics[width=1\textwidth,height=\textheight]{../../plot/NS_plot/10/fim_sub_reshape_0_9_t=43.png}

\subcaption{\label{}\(t=44\)}
\end{minipage}%

\caption{\label{fig-fim_NS}The evolution of the first row of FIM}

\end{figure}%

\subsection{Future Step}\label{future-step}

\begin{enumerate}
\def\labelenumi{\arabic{enumi}.}
\tightlist
\item
  TODO: Debug NS eigenvector and vjp.
\item
  TODO: Want to generate the full dataset for Francis' dataset (which
  might take 1 or 2 days).
\item
  TODO: Try it on Jason's dataset (Now that we fixed the problem with
  FIM computation, we are optimistic about the experiment, so we want to
  try it again.)
\end{enumerate}

\subsection{Question}\label{question}

\begin{enumerate}
\def\labelenumi{\arabic{enumi}.}
\tightlist
\item
  What would be the optimal number for observations, \(M\) when
  computing Fisher Information Matrix?
\end{enumerate}




\end{document}
